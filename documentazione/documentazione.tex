\documentclass[]{report}

\usepackage[a4paper, left=1.5cm, right=1.5cm, top=2.5cm, bottom=1.5cm]{geometry} % imposta le dimensioni della pagina
\usepackage{multicol}
\usepackage{tocloft}
\usepackage[]{lipsum}

\usepackage{hyperref} % permette agli hypertext di essere cliccabili
\hypersetup{
    pdftitle={Titolo PDF},
    pdfauthor={Autore PDF},
    colorlinks,
    linkcolor=black,
    citecolor=blue,
    filecolor=blue,
    urlcolor=blue
}

\usepackage{enumitem}
\setlist[itemize,enumerate]{
    topsep=0pt,
    itemsep=0pt,
    leftmargin=20pt
}
\usepackage[explicit]{titlesec}
\titleformat{\chapter}[display]{\Huge\bfseries}{\centering #1}{0pt}{}
% \renewcommand{\thesection}{\arabic{section}}
\titlespacing*{\subsection}{0pt}{30pt}{5pt}
\titlespacing*{\subsubsection}{0pt}{10pt}{2pt}

\usepackage{fancyhdr}
\pagestyle{fancy}
\fancypagestyle{plain}{%
  \renewcommand{\headrulewidth}{0pt}%
  \fancyhf{}%
}
\fancyfoot{}
\renewcommand{\chaptermark}[1]{\markboth{#1}{}}
\renewcommand{\sectionmark}[1]{\markright{#1}{}}
\fancyhead[L]{\leftmark}
\fancyhead[C]{\rightmark}
\fancyhead[R]{\thepage}
\newcommand{\sskip}{\smallskip \\ }
\newcommand{\meskip}{\medskip \\ }
\newcommand{\bskip }{\bigskip \\ }

\let\oldchapter\chapter
\renewcommand{\chapter}[1]{
    \newpage
    \begingroup%
    \let\clearpage\relax% Stop LaTeX from going to a new page
    \vspace*{\fill}%
    \oldchapter{#1}
    \vspace*{\fill}%
    \endgroup
    \thispagestyle{empty}
    \newpage
}
\let\oldsection\section
\renewcommand{\section}[1]{
    \newpage
    \oldsection{#1}
}

\linespread{1.2} % imposta lo spazio tra le righe
\fontdimen2\font=0.5em % imposta spazio tra le parole

\title{\textbf{Progetto LSO}\\Libreria}
\author{
    Ingenito Roberto - N86004077\\
    Ingenito Simone - N86004063\\
    Sequino Lorenzo - N86004367 
   }
\date{}


\begin{document}

\maketitle

\newpage

% \clearpage\pagestyle{empty}
% \begin{center}
% 	{\Huge\bfseries Indice\bigskip}
% \end{center}

% \setlength\columnsep{40pt}
% % \setlength{\columnseprule}{1pt} % divider
% \begin{multicols*}{2}
% 	\makeatletter
% 	\@starttoc{toc}
% 	\makeatother
% \end{multicols*}
% \clearpage\pagestyle{fancy}


\newpage
\chapter{Analisi dei requisiti}
\section{Obiettivo del progetto}
\section{Funzionalità principali}
\subsection{Registrazione e login utenti}
\subsection{Gestione del catalogo dei libri}
\subsection{Prestito di libri}
\subsection{Gestione del carrello e check-out}
\subsection{Notifiche agli utenti}

\chapter{Implementazione}
\section{Linguaggi di programmazione utilizzati}
C e SQL

\section{Struttura del codice}
scrivere come sono strutturate le cartelle, cosa c'è all'interno e il motivo di queste scelte



\section{Registrazione e login}
\subsection*{Client}
\subsection*{Server}

\section{Gestione dei libri e del catalogo}
\subsection*{Client}
\subsection*{Server}

\section{Prestito e restituzione libri}
\subsection*{Client}
\subsection*{Server}

\section{Carrello e check-out}
\subsection*{Client}
\subsection*{Server}

\section{Gestione delle notifiche}
\subsection*{Client}
\subsection*{Server}





\chapter{Database}
\section{Schema}
\begin{tikzpicture}
	% User table
	\begin{class}[text width=6cm]{user}{0,-7}
		\attribute{username: TEXT}
		\attribute{passw: TEXT}
	\end{class}

	% Book table
	\begin{class}[text width=7cm]{book}{10,-7}
		\attribute{isbn: TEXT}
		\attribute{title: TEXT}
		\attribute{authors: TEXT[]}
		\attribute{genre: TEXT[]}
		\attribute{quantity: INTEGER}
	\end{class}

	% Loan table
	\begin{class}[text width=8cm]{loan}{5,0}
		\attribute{id: SERIAL}
		\attribute{loan\_start: TIMESTAMP WITH TIME ZONE}
		\attribute{loan\_end: TIMESTAMP WITH TIME ZONE}
		\attribute{returned: TIMESTAMP WITH TIME ZONE}
		\attribute{username: TEXT}
		\attribute{isbn: TEXT}
	\end{class}

	% Relationships
	\association{user}{}{~\qquad0..*}{loan}{1..1\qquad~}{}
	\association{book}{}{0..*\qquad~}{loan}{~\qquad1..1}{}
\end{tikzpicture}

\section{Descrizione delle tabelle}
\section{Query principali utilizzate}


\end{document}




